\documentclass[a4paper, 10pt]{article}
%librerie / userpackage
\usepackage[italian]{babel}
\usepackage[T1]{fontenc}
\usepackage[utf8]{inputenc}
\usepackage{amsmath}
\usepackage{amsfonts}
\usepackage{amssymb}
\usepackage{calrsfs}
\renewcommand{\vec}{\mathbf}
\usepackage{graphicx}
\usepackage{titlesec}
\usepackage{float}
%settaggi package
\graphicspath{ {./img/} }
\setcounter{secnumdepth}{4}

\titleformat{\paragraph}
{\normalfont\normalsize\bfseries}{\theparagraph}{1em}{}
\titlespacing*{\paragraph}
{0pt}{3.25ex plus 1ex minus .2ex}{1.5ex plus .2ex}
%documento
\begin{document}
	\clearpage
	
	%pagina iniziale
	\begin{titlepage}
		\centering
		\vspace{\fill}
		{\scshape\LARGE Università degli Studi di Verona \par}
		\vspace{2cm} 
		
		\hrule
		\vspace{0.5cm}
		{\huge\bfseries Ingegneria del Software \par}
		\vspace{0.5cm}
		\hrule
		\vspace{1.5cm}
		
		{\LARGE\itshape Matteo Marzio VR421516 \par}
		\vspace{1.0cm}
		Documentazione Progetto Esercizio "Libreria" \par
		\vspace{1.0cm}
		Luglio - Settembre 2019 \par
		A.A 2018/2019\par

		\vspace{\fill}
	\end{titlepage}
	\tableofcontents
	%Inizio 
	\newpage
	\section{Descrizione generale e requisiti di sistema}
	Il sistema informatico prodotto si occupa nella gestione on-line di una libreria. 
	Tutti gli utenti vengono autenticati dal sistema per poter accedere alle loro funzionalità di
    competenza. Gli utenti posso registrarsi al sistema. In tal caso, gli utenti registrati avranno a
    disposizione una carta virtuale, che raccoglie i punti acquisiti dagli acquisti effettuati,
    denominata "LibroCard". I responsabili della libreria, oltre ad avere le stesse funzionalità 
    degli utenti registrati, potranno modificare, aggiungere e togliere i libri disponibili nel 
    negozio virtuale, modificare la classifica settimanale in base al genere dei libri e 
    controllare gli utenti e le loro LibroCard di referenza.
	\subsection{Specifiche dei casi d'uso}
		Gli attori protagonisti del sistema sono gli Utenti Standard, gli Utenti 
		Registrati e gli Amministratori. Tutti e tre visualizzano una stessa schermata 
		iniziale, denominata "Home". Successivamente gli utenti potranno decidere se 
		registrarsi, immettendo i propri dati personali, oppure accedere facendosi autenticare 
		dal sistema.
		\subsection{Utente Standard}
			Accede direttamente alla interfaccia di base; puo' visualizzare il catalogo dei 
			libri disponibili, controllare la classifica sia generale sia in base al genere 
			letterario, effettuare la registrazione per poter ottenere più funzionalità 
			da parte del sistema, ricercare i propri ( o degli altri utenti) ordini previo 
			invio del codice ordine e farsi autenticare dal sistema
			\begin{figure}[h]
				\centering
				\includegraphics[width=0.9\textwidth]{casi_uso/utente_standard.png}
				\caption{Casi d'uso dell'Utente Standard}
			\end{figure}
		\newpage
			\subsubsection{Ricerca Ordine}
			\fbox{\begin{minipage}{\dimexpr\textwidth-2\fboxsep-2\fboxrule\relax}
			\textbf{Attori} : Utente Standard, Utente Registrato, Amministratore \\
			\textbf{Precondizioni} : Nessuna \\
			\textbf{Passi} : 
				\begin{enumerate}
					\item L'utente accede al sistema
					\item L'utente visualizza l'interfaccia di base
					\item L'utente accede nella interfaccia per la visione degli ordini
					\item L'utente inserisce il codice ordine
					\item L'utente visiona l'ordine
				\end{enumerate}
			\textbf{Postcondizioni} : Nessuna
			\end{minipage}}
			\subsubsection{Visione della Classifica}
			\fbox{\begin{minipage}{\dimexpr\textwidth-2\fboxsep-2\fboxrule\relax}
			\textbf{Attori} : Utente Standard, Utente Registrato, Amministratore \\
			\textbf{Precondizioni} : Nessuna \\
			\textbf{Passi} : 
				\begin{enumerate}
					\item L'utente accede al sistema
					\item L'utente visualizza l'interfaccia di base
					\item L'utente accede nella interfaccia per la visione della classifica	
					\begin{enumerate}
						\item L'utente inserisce il genere desiderato 
						\item Il sistema mostra la classifica in base al genere
					\end{enumerate}
					\item L'utente visiona la classifica
				\end{enumerate}
			\textbf{Postcondizioni} : Nessuna
			\end{minipage}}
			\subsubsection{Visione del Catalogo}
			\fbox{\begin{minipage}{\dimexpr\textwidth-2\fboxsep-2\fboxrule\relax}
			\textbf{Attori} : Utente Standard, Utente Registrato, Amministratore \\
			\textbf{Precondizioni} : Nessuna \\
			\textbf{Passi} : 
				\begin{enumerate}
					\item L'utente accede al sistema
					\item L'utente visualizza l'interfaccia di base
					\item L'utente accede nella interfaccia per la visione del catalogo	
					\item L'utente visiona il catalogo.
				\end{enumerate}
			\textbf{Postcondizioni} : Nessuna
			\end{minipage}}
			
			\subsubsection{Registrazione}
			\fbox{\begin{minipage}{\dimexpr\textwidth-2\fboxsep-2\fboxrule\relax}
			\textbf{Attori} : Utente Standard \\
			\textbf{Precondizioni} : Nessuna \\
			\textbf{Passi} : 
				\begin{enumerate}
					\item L'utente accede al sistema
					\item L'utente visualizza l'interfaccia di base
					\item L'utente accede nella interfaccia per la registrazione	
					\item L'utente immette i propri dati personali
					\item L'utente conferma la registrazione
				\end{enumerate}
			\textbf{Postcondizioni} : Il sistema deve completare correttamente la registrazione
			\end{minipage}}
			
			\subsubsection{Autenticazione}
			\fbox{\begin{minipage}{\dimexpr\textwidth-2\fboxsep-2\fboxrule\relax}
			\textbf{Attori} : Utente Standard \\
			\textbf{Precondizioni} : L'utente Standard deve essere gia' registrato \\
			\textbf{Passi} : 
				\begin{enumerate}
					\item L'utente accede al sistema
					\item L'utente visualizza l'interfaccia di base
					\item L'utente accede nella interfaccia per l'autenticazione	
					\item L'utente immette i propri dati personali
					\item L'utente conferma l'accesso
				\end{enumerate}
			\textbf{Postcondizioni} : Il sistema deve autenticare correttamente l'utente
			\end{minipage}}
			\newpage
			\subsubsection{Diagramma di Attività: Utente Standard}
			\begin{figure}[h]
				\centering
				\includegraphics[width=1.2\textwidth]{activity_diagram/utente_standard.png}
				\caption{Activity Diagram : Utente Standard}
			\end{figure}
			\clearpage
		\newpage
		\subsection{Utente Registrato}
			Ha le stesse funzionalità dell'utente standard, ma essendo autenticato dal 
			sistema ha ulteriori caratteristiche. 
			Al primo accesso il sistema richiede l'aggiornamento delle informazioni personali.
			Puo' aggiungere i libri disponibili nel catalogo al carrello, 
			visionare direttamente (senza inserire il 
			codice ordine) i propri ordini, gestire il proprio carrello virtuale 
			(dove può proseguire con l'acquisto se volesse, oppure rimuovere i libri aggiunti), 
			modificare le proprie informazioni personali e disconnettersi dal sistema.
			\begin{figure}[h]
				\centering
				\includegraphics[width=1.2\textwidth]{casi_uso/utente_registrato.png}
				\caption{Casi d'uso dell'Utente Registrato}
			\end{figure}
		\subsubsection{Visione del Profilo}
			\fbox{\begin{minipage}{\dimexpr\textwidth-2\fboxsep-2\fboxrule\relax}
			\textbf{Attori} : Utente Registrato, Amministratore \\
			\textbf{Precondizioni} : Nessuna \\
			\textbf{Passi} : 
				\begin{enumerate}
					\item L'utente accede al sistema
					\item L'utente visualizza l'interfaccia di base
					\item L'utente accede nella interfaccia per la visione del catalogo	
					\item L'utente accede al profilo personale
					\begin{enumerate}
						\item L'utente visiona i propri punti "LibroCard"
						\item L'utente modifica i dati personali
						\item L'utente visiona i dati personali
					\end{enumerate}
				\end{enumerate}
			\textbf{Postcondizioni} : Nessuna
			\end{minipage}}
		\subsubsection{Gestione del Carrello}
			\fbox{\begin{minipage}{\dimexpr\textwidth-2\fboxsep-2\fboxrule\relax}
			\textbf{Attori} : Utente Registrato, Amministratore \\
			\textbf{Precondizioni} : L'utente deve aver aggiunto almeno un libro al carrello \\
			\textbf{Passi} : 
				\begin{enumerate}
					\item L'utente accede al sistema
					\item L'utente visualizza l'interfaccia di base
					\item L'utente accede nella interfaccia per la visione del carrello	
					\begin{enumerate}
						\item L'utente visiona i libri contenuti nel carrello
						\item L'utente rimuove i libri dal carrello
						\item L'utente procede all'acquisto, creando un ordine
					\end{enumerate}
				\end{enumerate}
			\textbf{Postcondizioni} : Il sistema deve eliminare correttamente i libri 
			selezionati; Il sistema deve creare correttamente l'ordine in base ai libri 
			contenuti nel carrello
			\end{minipage}}
		\subsubsection{Visione degli Ordini}
			\fbox{\begin{minipage}{\dimexpr\textwidth-2\fboxsep-2\fboxrule\relax}
			\textbf{Attori} : Utente Registrato, Amministratore \\
			\textbf{Precondizioni} : L'utente deve aver completato almeno un ordine \\
			\textbf{Passi} : 
				\begin{enumerate}
					\item L'utente accede al sistema
					\item L'utente visualizza l'interfaccia di base
					\item L'utente accede nella interfaccia per la visione degli ordini	
					\item Il sistema trova gli ordini appartenenti all'utente 
					\item Il sistema mostra gli ordini effettuati
					\item L'utente visiona gli ordini
				\end{enumerate}
			\textbf{Postcondizioni} : Nessuna
			\end{minipage}}
		\subsubsection{Conferma Ordine}
			\fbox{\begin{minipage}{\dimexpr\textwidth-2\fboxsep-2\fboxrule\relax}
			\textbf{Attori} : Utente Registrato, Amministratore \\
			\textbf{Precondizioni} : L'utente deve aver creato un ordine  \\
			\textbf{Passi} : 
				\begin{enumerate}
					\item L'utente controlla i libri che si vogliono acquistare
					\item L'utente sceglie la modalità di spedizione
					\item L'utente sceglia la modalità di pagamento
					\item L'utente immette, nel caso, un indirizzo secondario
					\item L'utente conferma l'ordine
				\end{enumerate}
			\textbf{Postcondizioni} : Il sistema deve completare l'ordine correttamente
			\end{minipage}}
		\newpage
		\subsubsection{Diagramma di Attività: Utente Registrato}
		\begin{figure}[h]
				\centering
				\includegraphics[width=0.85\textwidth]{activity_diagram/utente_registrato.png}
				\caption{Activity Diagram : Utente Registrato}
		\end{figure}
		\clearpage 
		%utile per gestire le immagini che non vogliono stare in un 
		%determinato spazio allocato
		\newpage
		\subsection{Amministratore}
		Ha le stesse funzionalità dell'utente registrato, ma facendo parte di un gruppo "ristretto"
		ha caratteristiche non disponibili per l'utente registrato. Può visionare
		gli utenti registrati (profilo, email, password, punti disponibili), 
		il catalogo (libri, prezzi, punti ottenibili, immagine associata) e la classifica 
		(modificabile manualmente). L'amministratore può aggiungere o togliere i permessi di admin 
		agli utenti registrati. Tutte queste funzionalità sono disponibili in una sezione diversa 
		del sito, con grafiche diverse per gestire al meglio (senza distrazioni) i dati salvati.
		\begin{figure}[h]
				\centering
				\includegraphics[width=1.2\textwidth]{casi_uso/amministratore.png}
				\caption{Casi d'uso dell'Utente Amministratore}
		\end{figure}
		\clearpage
		\subsubsection{Gestione Ordini}
			\fbox{\begin{minipage}{\dimexpr\textwidth-2\fboxsep-2\fboxrule\relax}
			\textbf{Attori} : Amministratore \\
			\textbf{Precondizioni} : L'utente deve aver fatto l'accesso 
			al sistema e deve essere autenticato dal sistema come amministratore \\
			\textbf{Passi} : 
				\begin{enumerate}
					\item L'utente visualizza l'interfaccia di base
					\item L'utente accede alla pagina dedicata per gli amministratori
					\item L'utente accede alla sezione degli ordini
					\item L'utente visiona gli ordini
					\item L'utente modifica un ordine
						\begin{enumerate}
							\item L'utente cambia l'indirizzo secondario di spedizione 
							\item L'utente cambia lo stato della spedizione
							\item L'utente cancella l'ordine
						\end{enumerate}
				\end{enumerate}
			\textbf{Postcondizioni} : Il sistema deve completare le modifiche, se presenti,
			 correttamente
			\end{minipage}}
		\subsubsection{Gestione Classifiche}
			\fbox{\begin{minipage}{\dimexpr\textwidth-2\fboxsep-2\fboxrule\relax}
			\textbf{Attori} : Amministratore \\
			\textbf{Precondizioni} : L'utente deve aver fatto l'accesso 
			al sistema e deve essere autenticato dal sistema come amministratore \\
			\textbf{Passi} : 
				\begin{enumerate}
					\item L'utente visualizza l'interfaccia di base
					\item L'utente accede alla pagina dedicata per gli amministratori
					\item L'utente accede alla sezione delle classifiche
					\item L'utente visiona le classifiche presenti
						\begin{enumerate}
							\item L'utente modifica posizione e/o genere del libro in classifica
							\item L'utente cancella il libro dalla classifica di riferimento
						\end{enumerate}
				\end{enumerate}
			\textbf{Postcondizioni} : Il sistema deve completare le modifiche, se presenti,
			 correttamente
			\end{minipage}}
		\subsubsection{Gestione Libri}
			\fbox{\begin{minipage}{\dimexpr\textwidth-2\fboxsep-2\fboxrule\relax}
			\textbf{Attori} : Amministratore \\
			\textbf{Precondizioni} : L'utente deve aver fatto l'accesso 
			al sistema e deve essere autenticato dal sistema come amministratore \\
			\textbf{Passi} : 
				\begin{enumerate}
					\item L'utente visualizza l'interfaccia di base
					\item L'utente accede alla pagina dedicata per gli amministratori
					\item L'utente accede alla sezione dei libri
					\item L'utente visiona i libri presenti
					\begin{enumerate}
						\item L'utente modifica un libro
						\begin{enumerate}
							\item L'utente modifica i dati del libro (compresa 
							l'immagine di riferimento)
							\item L'utente rimuove il libro dalla classifica
						\end{enumerate}
						\item L'utente aggiunge un libro, inserendo i dati di riferimento corretti
					\end{enumerate}
				\end{enumerate}
			\textbf{Postcondizioni} : Il sistema deve completare le modifiche, se presenti,
			 correttamente
			\end{minipage}}
		\subsubsection{Gestione Utenti}
			\fbox{\begin{minipage}{\dimexpr\textwidth-2\fboxsep-2\fboxrule\relax}
			\textbf{Attori} : Amministratore \\
			\textbf{Precondizioni} : L'utente deve aver fatto l'accesso 
			al sistema e deve essere autenticato dal sistema come amministratore \\
			\textbf{Passi} : 
				\begin{enumerate}
					\item L'utente visualizza l'interfaccia di base
					\item L'utente accede alla pagina dedicata per gli amministratori
					\item L'utente accede alla sezione degli utenti
					\item L'utente visiona gli utenti presenti
					\item L'admin modifica un utente
						\begin{enumerate}
							\item L'admin modifica i dati dell'utente registrato
							\item L'admin elimina l'utente registrato
							\item L'admin aggiunge un utente, inserendo i dati di riferimento corretti
							\item L'admin concede o toglie i diritti di amministratore 
							all'utente registrato
						\end{enumerate}
				\end{enumerate}
			\textbf{Postcondizioni} : Il sistema deve completare le modifiche, se presenti,
			 correttamente
			\end{minipage}}
		\subsubsection{Gestione LibroCard}
			\fbox{\begin{minipage}{\dimexpr\textwidth-2\fboxsep-2\fboxrule\relax}
			\textbf{Attori} : Amministratore \\
			\textbf{Precondizioni} : L'utente deve aver fatto l'accesso 
			al sistema e deve essere autenticato dal sistema come amministratore \\
			\textbf{Passi} : 
				\begin{enumerate}
					\item L'utente visualizza l'interfaccia di base
					\item L'utente accede alla pagina dedicata per gli amministratori
					\item L'utente accede alla sezione delle LibroCard
					\item L'utente visiona le carte disponibili
					\item L'admin modifica una carta
						\begin{enumerate}
							\item L'admin modifica i punti 
							\item L'admin elimina la carta dell'utente di riferimento
						\end{enumerate}
				\end{enumerate}
			\textbf{Postcondizioni} : Il sistema deve completare le modifiche, se presenti,
			 correttamente
			\end{minipage}}
		\newpage
		\subsubsection{Diagramma di Attività: Utente Amministratore}
		\begin{figure}[h]
				\centering
				\includegraphics[width=0.76\textwidth]{activity_diagram/amministratore_1.png}
				\caption{Activity Diagram : Amministratore (Parte 1)}
		\end{figure} \clearpage \newpage
		\begin{figure}[h]
				\centering
				\includegraphics[width=1.0\textwidth]{activity_diagram/amministratore_2.png}
				\caption{Activity Diagram : Amministratore (Parte 2)}
		\end{figure} \clearpage 
		\newpage
	\section{Sviluppo: progetto della architettura ed implementazione del sistema}
	\subsection{Note sul processo di Sviluppo}
		Il processo di sviluppo e' stato un ibrido tra il metodo Plan-Driven e Agile. 
		\\Le attività principali, quali la gestione degli utenti e il sito web, sono state
		 pianificate prima di essere progettate e implementate ma allo stesso modo 
		 per le parti piu' specifiche 
		si e' usato un metodo di pianificazione incrementale, aventi pero' meno priorità 
		rispetto alla costruzione della struttura principale. Alla base del progetto e' 
		applicata l'idea di progettazione prima della implementazione, con elevata 
		possibilità di refactoring sul codice già scritto, in quanto e' importante 
		avere una base solida su cui 
		lavorare e una linea guida da seguire, soprattutto per i programmatori/tecnici 
		esterni al progetto. \\ 
		Dopo aver effettuato un'attenta analisi dei requisiti 
		di sistema sono stati creati i casi d'uso,
		con relative specifiche per definire le azioni che 
		possono svolgere (o meno) gli attori coinvolti 
		(in questo caso Utente Standard, Utente Registrato e Amministratore). 
		Tuttavia la documentazione e la creazione dei diagrammi di 
		attività sono state create successivamente alla rifinitura del
		progetto, in quanto e' sempre possibile scoprire nuove funzionalità per 
		gli utenti, ma principalmente per mettere in chiaro i passaggi svolti e 
		confermare o meno i casi d'uso degli attori. \\
		Le scelte architetturali sono state stabilite in precedenza all'implementazione 
		per esperienza personale (avendo già lavorato con Django in precedenza. 
	\subsection{Progettazione del sistema e pattern architetturali}
	Il sistema e' stato progettato utilizzando le tecniche di modellazione ad oggetti 
	implementate in un web-framework basato su Python. Questa scelta e' stata 
	utilizzata per tre motivi: la prima motivazione e' basata sulla popolarita' 
	che stanno avendo le applicazioni e servizi web rispetto alle applicazioni 
	tradizionali (e quindi servizio accessibile ovunque), la seconda e' stata 
	l'influenza basata sulle personali esperienze passate a lavorare con questo 
	web-framework, in quanto Django, basato su Python, permette di implementare 
	le migliaia di librerie disponibili sul Web e implementarle nei siti web. 
	L'ultimo motivo, il piu' importante, e' l'implementazione nativa del pattern 
	MVC, ovvero Model-View-Control. Grazie a cio', e' stato semplice pensare 
	e progettare il sistema come un blocco formato da tre parti e quindi 
	gestire al meglio la programmazione.
	\begin{itemize}
		\item \textbf{Model:} riguarda i dati e le informazioni salvate. 
		Definisce principalmente come i dati sono immagazzinati, con quale formato
	    e la loro struttura.
		\item \textbf{View:} rappresenta visivamente il modello e quindi 
		anche i dati del sistema. Django permette di implementare il codice 
		Python all'interno delle pagine HTML per creare la parte "Vista".
		\item \textbf{Control}: definisce la logica applicativa, 
		ovvero il comportamento del sistema rispetto agli stimoli 
		esterni. Consiste nell'ascolto delle azioni compiute nel View.
	\end{itemize} Elemento interessante nel lavorare in Django e' 
		che si può optare per una modellazione MVT, ovvero Model-View-Template. Difatti, Django stesso
		si occupa della parte Controller, lasciando i Template (parte grafica) al programmatore. 
		Per gestire i vari controllori, Django usa gli URL per gestire al meglio quale View 
		utilizzare. L'URL in questo caso agisce come "ActionListener" della parte controller.
	
		\begin{figure}[h]
				\centering
				\includegraphics[width=0.9\textwidth]{diagrammi/mvc.png}
				\caption{Diagramma della architettura MVC tradizionale}
		\end{figure} 
		\begin{figure}[h]
				\centering
				\includegraphics[width=0.9\textwidth]{diagrammi/mvt.png}
				\caption{Diagramma della architettura MVT riferita a Django}
		\end{figure} \clearpage
		
	\subsubsection{Diagramma delle classi del modello }	
		
		\begin{figure}[h]
				\centering
				\includegraphics[width=1.2\textwidth]{diagrammi/models.png}
				\caption{Diagramma delle classi del modello}
		\end{figure}\clearpage 
		
	\subsubsection{Diagramma della relazione tra View e Controller }	
		\begin{figure}[h]
				\centering
				\includegraphics[width=1.2\textwidth]{diagrammi/view_controller.png}
				\caption{Diagramma relazione View - Controller}
		\end{figure}\clearpage 
	
	\subsection{Progettazione delle Componenti}
	In Django, come detto in precedenza, gestisce personalmente il controller in modo tale
	da permettere di concentrarsi direttamente sulle View e sui Template. 
	Tuttavia, nella figura precedente, e' rappresentata una possibile iterazione 
	tra View e Controller, che segue il pattern Observer. Il patter Observer 
	si basa sul concetto di soggetto che cambia stato e osservatore che viene 
	avvisato del cambiamento. Quando avviene un cambiamento nello stato dell'oggetto, 
	gli osservatori se ne accorgono e vengono aggiornati di conseguenza. \\
	Nel template Home (che implementa la sezione navbar) e' presente un ascoltatore che reagisce 
	a seconda dell'utente che si presenta.  \\ Il template Catalog (o Library) viene 
	ascoltato dal controllore per la ricerca di un libro nella libreria e l'
	inserimento dei libri da acquistare. 
	\\ Il template Registration si occupa della registrazione dell'utente chiedendo ad esso 
	nome utente e password desiderata. 
	Il controllore si occuperà nel controllare che i dati siano corretti e nel caso si
	occuperà anche di aggiungere questi dati nel database. In caso contrario all'utente 
	verrà mostrato un errore con refresh della pagina
	\\ Il template Login si occupa della autenticazione dell'utente chiedendo nome utente 
	e password. Il controllore si occuperà nel controllare che i dati siano corretti e in caso positivo 
	controllerà che i dati siano corretti. In caso contrario all'utente 
	verrà mostrato un errore con refresh della pagina. 
	Inoltre il navbar si aggiornerà di conseguenza mostrando elementi accessibili solo agli utenti
	registrati.
	\\ Il template Profile viene attivato immediatamente al primo accesso di un utente 
	registrato. L'utente per completare la registrazione dovrà inserire i dati personali. 
	Il controllore si occuperà nel controllare che i dati siano corretti e in caso positivo 
	controllerà che i dati siano corretti. In caso contrario all'utente 
	verrà mostrato un errore con refresh della pagina. Successivamente l'utente può sempre modificare
	i propri dati personali in qualunque momento, e il controllore gestita le modifiche 
	\\ L'utente, dopo essersi registrato, potrà aggiungere i libri nel carrello. Il template Cart 
	mostrerà all'utente i libri acquisibili aggiunti al carrello e, 
	se l'utente vorrà, potrà procedere all'acquisto. 
	Il controllore gestirà il passaggio dal template Cart a quello dell'Order, 
	controllando che tutti i libri siano disponibili e che non ci siano errori durante questa fase. 
	Quando all'utente si presenterà il template Order, egli potra decidere se inserire 
	un secondo indirizzo di spedizione e la metodologia di pagamento. 
	Ovviamente il template mostrerà per l'ultima volta i libri che si vogliono acquistare.
	\newpage
	\subsection{Diagrammi di Sequenza}
	I diagrammi di sequenza servono a specificare il funzionamento e la sequenza 
	delle operazioni che avvengono durante l'esecuzione del sistema e specificano 
	gli eventi che accadono durante queste esecuzioni, compresi i messaggi scambiati. 
	\begin{figure}[h]
				\centering
				\includegraphics[width=1.0\textwidth]{sequence_diagram/registrazione.png}
				\caption{Sequence Diagram: Registrazione}
	\end{figure}
	\begin{figure}[h]
				\centering
				\includegraphics[width=1.2\textwidth]{sequence_diagram/login.png}
				\caption{Sequence Diagram: Login}
	\end{figure}
	\begin{figure}[h]
				\centering
				\includegraphics[width=1.0\textwidth]{sequence_diagram/ordine.png}
				\caption{Sequence Diagram: Ordine}
	\end{figure}\clearpage
	\newpage
	\subsection{Test e Validazione}
	La fase di test e' servita per verificare le varie componenti del sistema 
	e per accertarsi che ognuna svola i suoi obiettivi correttamente. A livello statico si e' effettuata    
	una ispezione di diversi fattori come i requisiti, l'architettura e il codice effettivo, 
	mentre a livello dinamico sono stati in atto test di sviluppo partendo dai singoli 
	componenti per poi espandersi all'intero sistema. Questa modalità di controllo e' 
	stata molto utile per individuare errori e bug del sistema (come ad esempio la 
	mancanza dei punti assegnati per i libri acquisiti, oppure componenti grafiche 
	sballate nel momento in cui si utilizzava una componente particolare). 
	\\ I test sono stati messi in atto sul sistema simulandone l'utilizzo 
	sia come utenti standard sia come utenti registrati che come 
	amministratori/responsabili. Inoltre, i test sono stati effettuati 
	anche da (pochi) utenti finali estranei al progetto, per individuare 
	i punti deboli del sistema da riprogettare o da sistemare. Tale modo 
	di effettuare le prove ha aiutato al deploy finale del progetto.
\end{document}